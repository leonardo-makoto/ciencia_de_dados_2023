% Options for packages loaded elsewhere
\PassOptionsToPackage{unicode}{hyperref}
\PassOptionsToPackage{hyphens}{url}
%
\documentclass[
]{article}
\usepackage{amsmath,amssymb}
\usepackage{lmodern}
\usepackage{iftex}
\ifPDFTeX
  \usepackage[T1]{fontenc}
  \usepackage[utf8]{inputenc}
  \usepackage{textcomp} % provide euro and other symbols
\else % if luatex or xetex
  \usepackage{unicode-math}
  \defaultfontfeatures{Scale=MatchLowercase}
  \defaultfontfeatures[\rmfamily]{Ligatures=TeX,Scale=1}
\fi
% Use upquote if available, for straight quotes in verbatim environments
\IfFileExists{upquote.sty}{\usepackage{upquote}}{}
\IfFileExists{microtype.sty}{% use microtype if available
  \usepackage[]{microtype}
  \UseMicrotypeSet[protrusion]{basicmath} % disable protrusion for tt fonts
}{}
\makeatletter
\@ifundefined{KOMAClassName}{% if non-KOMA class
  \IfFileExists{parskip.sty}{%
    \usepackage{parskip}
  }{% else
    \setlength{\parindent}{0pt}
    \setlength{\parskip}{6pt plus 2pt minus 1pt}}
}{% if KOMA class
  \KOMAoptions{parskip=half}}
\makeatother
\usepackage{xcolor}
\usepackage[margin=1in]{geometry}
\usepackage{color}
\usepackage{fancyvrb}
\newcommand{\VerbBar}{|}
\newcommand{\VERB}{\Verb[commandchars=\\\{\}]}
\DefineVerbatimEnvironment{Highlighting}{Verbatim}{commandchars=\\\{\}}
% Add ',fontsize=\small' for more characters per line
\usepackage{framed}
\definecolor{shadecolor}{RGB}{248,248,248}
\newenvironment{Shaded}{\begin{snugshade}}{\end{snugshade}}
\newcommand{\AlertTok}[1]{\textcolor[rgb]{0.94,0.16,0.16}{#1}}
\newcommand{\AnnotationTok}[1]{\textcolor[rgb]{0.56,0.35,0.01}{\textbf{\textit{#1}}}}
\newcommand{\AttributeTok}[1]{\textcolor[rgb]{0.77,0.63,0.00}{#1}}
\newcommand{\BaseNTok}[1]{\textcolor[rgb]{0.00,0.00,0.81}{#1}}
\newcommand{\BuiltInTok}[1]{#1}
\newcommand{\CharTok}[1]{\textcolor[rgb]{0.31,0.60,0.02}{#1}}
\newcommand{\CommentTok}[1]{\textcolor[rgb]{0.56,0.35,0.01}{\textit{#1}}}
\newcommand{\CommentVarTok}[1]{\textcolor[rgb]{0.56,0.35,0.01}{\textbf{\textit{#1}}}}
\newcommand{\ConstantTok}[1]{\textcolor[rgb]{0.00,0.00,0.00}{#1}}
\newcommand{\ControlFlowTok}[1]{\textcolor[rgb]{0.13,0.29,0.53}{\textbf{#1}}}
\newcommand{\DataTypeTok}[1]{\textcolor[rgb]{0.13,0.29,0.53}{#1}}
\newcommand{\DecValTok}[1]{\textcolor[rgb]{0.00,0.00,0.81}{#1}}
\newcommand{\DocumentationTok}[1]{\textcolor[rgb]{0.56,0.35,0.01}{\textbf{\textit{#1}}}}
\newcommand{\ErrorTok}[1]{\textcolor[rgb]{0.64,0.00,0.00}{\textbf{#1}}}
\newcommand{\ExtensionTok}[1]{#1}
\newcommand{\FloatTok}[1]{\textcolor[rgb]{0.00,0.00,0.81}{#1}}
\newcommand{\FunctionTok}[1]{\textcolor[rgb]{0.00,0.00,0.00}{#1}}
\newcommand{\ImportTok}[1]{#1}
\newcommand{\InformationTok}[1]{\textcolor[rgb]{0.56,0.35,0.01}{\textbf{\textit{#1}}}}
\newcommand{\KeywordTok}[1]{\textcolor[rgb]{0.13,0.29,0.53}{\textbf{#1}}}
\newcommand{\NormalTok}[1]{#1}
\newcommand{\OperatorTok}[1]{\textcolor[rgb]{0.81,0.36,0.00}{\textbf{#1}}}
\newcommand{\OtherTok}[1]{\textcolor[rgb]{0.56,0.35,0.01}{#1}}
\newcommand{\PreprocessorTok}[1]{\textcolor[rgb]{0.56,0.35,0.01}{\textit{#1}}}
\newcommand{\RegionMarkerTok}[1]{#1}
\newcommand{\SpecialCharTok}[1]{\textcolor[rgb]{0.00,0.00,0.00}{#1}}
\newcommand{\SpecialStringTok}[1]{\textcolor[rgb]{0.31,0.60,0.02}{#1}}
\newcommand{\StringTok}[1]{\textcolor[rgb]{0.31,0.60,0.02}{#1}}
\newcommand{\VariableTok}[1]{\textcolor[rgb]{0.00,0.00,0.00}{#1}}
\newcommand{\VerbatimStringTok}[1]{\textcolor[rgb]{0.31,0.60,0.02}{#1}}
\newcommand{\WarningTok}[1]{\textcolor[rgb]{0.56,0.35,0.01}{\textbf{\textit{#1}}}}
\usepackage{graphicx}
\makeatletter
\def\maxwidth{\ifdim\Gin@nat@width>\linewidth\linewidth\else\Gin@nat@width\fi}
\def\maxheight{\ifdim\Gin@nat@height>\textheight\textheight\else\Gin@nat@height\fi}
\makeatother
% Scale images if necessary, so that they will not overflow the page
% margins by default, and it is still possible to overwrite the defaults
% using explicit options in \includegraphics[width, height, ...]{}
\setkeys{Gin}{width=\maxwidth,height=\maxheight,keepaspectratio}
% Set default figure placement to htbp
\makeatletter
\def\fps@figure{htbp}
\makeatother
\setlength{\emergencystretch}{3em} % prevent overfull lines
\providecommand{\tightlist}{%
  \setlength{\itemsep}{0pt}\setlength{\parskip}{0pt}}
\setcounter{secnumdepth}{-\maxdimen} % remove section numbering
\ifLuaTeX
  \usepackage{selnolig}  % disable illegal ligatures
\fi
\IfFileExists{bookmark.sty}{\usepackage{bookmark}}{\usepackage{hyperref}}
\IfFileExists{xurl.sty}{\usepackage{xurl}}{} % add URL line breaks if available
\urlstyle{same} % disable monospaced font for URLs
\hypersetup{
  hidelinks,
  pdfcreator={LaTeX via pandoc}}

\author{}
\date{\vspace{-2.5em}}

\begin{document}

\hypertarget{mae-5905-introduuxe7uxe3o-uxe0-ciuxeancia-de-dados}{%
\section{MAE 5905: Introdução à Ciência de
Dados}\label{mae-5905-introduuxe7uxe3o-uxe0-ciuxeancia-de-dados}}

\hypertarget{prova-1.-primeiro-semestre-de-2023.-entregar-19052023.}{%
\subsection{Prova 1. Primeiro Semestre de 2023. Entregar
19/05/2023.}\label{prova-1.-primeiro-semestre-de-2023.-entregar-19052023.}}

Aluno: Leonardo Makoto Kawahara - 7180679

\begin{Shaded}
\begin{Highlighting}[]
\FunctionTok{set.seed}\NormalTok{(}\DecValTok{1}\NormalTok{)}
\FunctionTok{library}\NormalTok{(tidyverse)}
\FunctionTok{library}\NormalTok{(MASS)}
\FunctionTok{library}\NormalTok{(ISLR)}
\FunctionTok{library}\NormalTok{(caret)}
\FunctionTok{library}\NormalTok{(glmnet)}
\FunctionTok{library}\NormalTok{(ggplot2)}
\end{Highlighting}
\end{Shaded}

\hypertarget{exercuxedcio-1}{%
\section{Exercício 1}\label{exercuxedcio-1}}

\hypertarget{item-1a}{%
\subsection{Item 1a}\label{item-1a}}

Considere o caso de duas populações exponenciais, uma com média 1 e
outra com média 0,5. Supondo \(\pi_1\) = \(\pi_2\), encontre o
classificador de Bayes. Quais são as probabilidades de classificação
incorreta? Construa um gráfico, mostrando a fronteira de decisão e as
regiões de classificação em cada população. Generalize para o caso das
médias serem \(\alpha > 0\) e \(\beta > 0\), respectivamente.

\hypertarget{resposta}{%
\subsubsection{Resposta:}\label{resposta}}

Sejam \(X_1\) e \(X_2\) variáveis aleatórias com média 1 e média 0,5,
respectivamente. Temos que \begin{equation}
f(x_1;\lambda_1)=
\begin{cases}
\lambda_1 e^{-\lambda_1 x}, & \text{se}\ x_1 \geq 0 \\
0, & \text{caso contrário}
\end{cases}
\\
f(x_2;\lambda_2)=
\begin{cases}
\lambda_2 e^{-\lambda_2 x}, & \text{se}\ x_2 \geq 0 \\
0, & \text{caso contrário}
\end{cases}
\end{equation}

Como a \(E(X_1) = 1\) e \(E(X_2) = 0,1\), temos que \(\lambda_1 = 1\) e
\(\lambda_2 = 2\)

A probabilidade de classifição incorreta é dadao por: \begin{equation}
P(C_k | C_j, R) = \int_{\chi_k} f_j(x) dx
\end{equation}

Como temos duas classes, para elementos com valor das variáveis
preditoras igual a x devem ser classicada em \(C_1\) se

\begin{equation}
\frac{f_1(x)}{f_2(x)} \geq \frac{\pi_2}{\pi_1} = \frac{0,5}{0,5} = 1
\end{equation} e em \(C_2\), caso contrário.

Isto é, \begin{equation}
\frac{e^{-x}}{2e} \geq 1 \longrightarrow
e^x \geq 2 \longrightarrow
x \geq log2
\end{equation}

Logo, temos que, \begin{equation}
\chi_2= [0; log2] \\
\chi_1 = [log2;\infty] \\

\end{equation} Pela formula de probabilidade de classificação incorreta
\begin{equation}
P(C_1 | C_2; R) = \int^\infty_{log2}2e^{-2x}dx = 0.25\\
P(C_2 | C_1; R) = \int^{log2}_{0}e^{-x}dx = -e^{-log2} - (-e^0 ) = \frac{1}{2} \\

\end{equation}

Criando o gráfico de fronteiras

\begin{Shaded}
\begin{Highlighting}[]
\CommentTok{\# Gerando os dados de exemplo}

\CommentTok{\# Criando o gráfico}
\FunctionTok{ggplot}\NormalTok{() }\SpecialCharTok{+}

  \FunctionTok{geom\_vline}\NormalTok{(}\AttributeTok{xintercept =} \FunctionTok{log}\NormalTok{(}\DecValTok{2}\NormalTok{), }\AttributeTok{linetype =} \StringTok{"dashed"}\NormalTok{) }\SpecialCharTok{+}
  \FunctionTok{geom\_text}\NormalTok{(}\FunctionTok{aes}\NormalTok{(}\AttributeTok{x =} \DecValTok{2}\NormalTok{, }\AttributeTok{y =} \FloatTok{0.5}\NormalTok{, }\AttributeTok{label =} \StringTok{"Classe 1"}\NormalTok{), }\AttributeTok{hjust =} \DecValTok{0}\NormalTok{, }\AttributeTok{vjust =} \DecValTok{0}\NormalTok{) }\SpecialCharTok{+}
  \FunctionTok{geom\_text}\NormalTok{(}\FunctionTok{aes}\NormalTok{(}\AttributeTok{x =} \DecValTok{0}\NormalTok{, }\AttributeTok{y =} \FloatTok{0.5}\NormalTok{, }\AttributeTok{label =} \StringTok{"Classe 2"}\NormalTok{), }\AttributeTok{hjust =} \DecValTok{1}\NormalTok{, }\AttributeTok{vjust =} \DecValTok{0}\NormalTok{) }\SpecialCharTok{+}
  \FunctionTok{labs}\NormalTok{(}\AttributeTok{x =} \StringTok{"X"}\NormalTok{, }\AttributeTok{y =} \StringTok{"Y"}\NormalTok{, }\AttributeTok{title =} \StringTok{"Fronteira de Decisão e Origem"}\NormalTok{) }\SpecialCharTok{+}
  \FunctionTok{xlim}\NormalTok{(}\SpecialCharTok{{-}}\DecValTok{2}\NormalTok{, }\DecValTok{5}\NormalTok{)}
\end{Highlighting}
\end{Shaded}

\includegraphics{prova1_files/figure-latex/unnamed-chunk-1-1.pdf}

Generalizando, temos que \begin{equation}
f(x_1;\lambda_1 = \frac{1}{\alpha})=
\begin{cases}
\frac{1}{\alpha} e^{-\frac{1}{\alpha} x}, & \text{se}\ x_1 \geq 0 \\
0, & \text{caso contrário}
\end{cases}
\\
f(x_2;\lambda_2 = \frac{1}{\beta})=
\begin{cases}
\frac{1}{\beta}e^{-\frac{1}{\beta} x}, & \text{se}\ x_2 \geq 0 \\
0, & \text{caso contrário}
\end{cases}
\end{equation}

Como temos duas classes, para elementos com valor das variáveis
preditoras igual a x devem ser classicada em \(C_1\) se

\begin{equation}
\frac{f_1(x)}{f_2(x)} \geq \frac{\pi_2}{\pi_1} = \frac{0,5}{0,5} = 1
\end{equation}

Isto é, \begin{equation}
\frac{\frac{1}{\alpha} e^{-\frac{1}{\alpha} x}}{\frac{1}{\beta}e^{-\frac{1}{\beta} x}} \geq 1 \longrightarrow
e^{x(\frac{\alpha - \beta}{\alpha\beta})} \geq \frac{\alpha}{\beta} \longrightarrow
x \geq log(\frac{\alpha}{\beta})\frac{\alpha \beta}{\alpha-\beta}
\end{equation}

Então, temos que \begin{equation}
\chi_1 = [log(\frac{\alpha}{\beta})\frac{\alpha \beta}{\alpha-\beta}; \infty] \\
\chi_2 = [0;log(\frac{\alpha}{\beta})\frac{\alpha \beta}{\alpha-\beta}] \\

\end{equation}

Pela formula de probabilidade de classificação incorreta
\begin{equation}
P(C_1 | C_2; R) = \int^\infty_{log(\frac{\alpha}{\beta})\frac{\alpha \beta}{\alpha-\beta}}\frac{1}{\beta}e^{-\frac{1}{\beta} x}dx \\
P(C_2 | C_1; R) = \int^{log(\frac{\alpha}{\beta})\frac{\alpha \beta}{\alpha-\beta}}_{0}\frac{1}{\alpha} e^{-\frac{1}{\alpha} x}dx  \\


\end{equation}

\hypertarget{item-1b}{%
\subsection{Item 1b}\label{item-1b}}

Simule 200 observações de cada distribuição exponencial da parte (a).
Usando os dados para estimar os parâmetros, supostos agora
desconhecidos, obtenha o classificador de Bayes, a fronteira de decisão
e as probabilidades de classificação incorreta com a regra obtida no
exercício anterior. Compare os resultados com aqueles obtidos no item
(a).

\hypertarget{resposta-1}{%
\subsubsection{Resposta:}\label{resposta-1}}

Simulando cada distribuição exponecial

\begin{Shaded}
\begin{Highlighting}[]
\NormalTok{lamda\_1 }\OtherTok{\textless{}{-}} \DecValTok{1}
\NormalTok{lambda\_2 }\OtherTok{\textless{}{-}} \DecValTok{2}
\NormalTok{n }\OtherTok{\textless{}{-}} \DecValTok{200}

\NormalTok{amostra\_1 }\OtherTok{\textless{}{-}} \FunctionTok{rexp}\NormalTok{(n, }\AttributeTok{rate =} \DecValTok{1}\SpecialCharTok{/}\NormalTok{lamda\_1)}
\NormalTok{amostra\_2 }\OtherTok{\textless{}{-}} \FunctionTok{rexp}\NormalTok{(n, }\AttributeTok{rate =} \DecValTok{1}\SpecialCharTok{/}\NormalTok{lambda\_2)}
\end{Highlighting}
\end{Shaded}

Vamos assumir que os parâmetros são desconhecidos Nesse contexto, vamos
assumir que \(X_j|Y = k \sim exp(\lambda_i)\). Temos que estmiar a média
e variância para cada amostra:

\begin{Shaded}
\begin{Highlighting}[]
\NormalTok{media\_1 }\OtherTok{\textless{}{-}}  \FunctionTok{mean}\NormalTok{(amostra\_1)}

\NormalTok{media\_2 }\OtherTok{\textless{}{-}}  \FunctionTok{mean}\NormalTok{(amostra\_2)}
\end{Highlighting}
\end{Shaded}

Temos que \(\mu_1 = 0.9999842\) e \(mu_2 = 1.8925848\). Como temos duas
classes, para elementos com valor das variáveis preditoras igual a x
devem ser classicada em \(C_1\) se

\begin{equation}
\frac{f_1(x)}{f_2(x)} \geq \frac{\pi_2}{\pi_1} = \frac{0,5}{0,5} = 1 \longrightarrow x \leq 1.3526

\end{equation}

A fronteira de classificação, portanto, é 1.3526

\hypertarget{exercuxedcio-2}{%
\section{Exercício 2}\label{exercuxedcio-2}}

Considere os dados do arquivo \texttt{disco} e a variável resposta
\texttt{y} = 1 se o disco estiver deslocado e \texttt{y} = 0, caso
contrário. Use a função discriminante linear de Fisher para obter um
classificador. Tome o conjunto de treinamento aquele contendo as
primeiras 80 observações e o conjunto de teste contendo as demais 24
observações. Obtenha um classificador tendo como variável preditora a
distância aberta e outro tendo como preditores as duas distâncias. Use a
função \texttt{lda()} do pacote \texttt{MASS}. Interprete os resultados
e escolha o melhor classificador usando a acurácia como base. Obtenha a
sensibilidade e especificidade de cada classificador.

\hypertarget{resposta-2}{%
\subsubsection{Resposta:}\label{resposta-2}}

Extraindo o arquivo

\begin{Shaded}
\begin{Highlighting}[]
\NormalTok{disco }\OtherTok{\textless{}{-}}\NormalTok{ readxl}\SpecialCharTok{::}\FunctionTok{read\_excel}\NormalTok{(}\StringTok{"disco.xls"}\NormalTok{)}
\FunctionTok{head}\NormalTok{(disco)}
\end{Highlighting}
\end{Shaded}

\begin{verbatim}
## # A tibble: 6 x 3
##   deslocamento distanciaA distanciaF
##          <dbl>      <dbl>      <dbl>
## 1            0        2.2        1.4
## 2            0        2.4        1.2
## 3            0        2.6        2  
## 4            1        3.5        1.8
## 5            0        1.3        1  
## 6            1        2.8        1.1
\end{verbatim}

Definindo as amostras de treino e teste

\begin{Shaded}
\begin{Highlighting}[]
\NormalTok{treino }\OtherTok{\textless{}{-}}\NormalTok{ disco[}\DecValTok{1}\SpecialCharTok{:}\DecValTok{80}\NormalTok{,]}
\NormalTok{teste }\OtherTok{\textless{}{-}}\NormalTok{ disco[}\DecValTok{81}\SpecialCharTok{:}\DecValTok{104}\NormalTok{,]}
\end{Highlighting}
\end{Shaded}

Estimando classificador com preditor distância aberta

\begin{Shaded}
\begin{Highlighting}[]
\NormalTok{lda\_dist\_aberta }\OtherTok{\textless{}{-}} \FunctionTok{lda}\NormalTok{(deslocamento }\SpecialCharTok{\textasciitilde{}}\NormalTok{ distanciaA, }\AttributeTok{data =}\NormalTok{ treino)}
\NormalTok{lda\_pred\_dist\_aberta }\OtherTok{\textless{}{-}} \FunctionTok{predict}\NormalTok{(lda\_dist\_aberta, }\AttributeTok{newdata =}\NormalTok{ teste)}
\end{Highlighting}
\end{Shaded}

Estimando classificador cujas preditoras são as duas distâncias

\begin{Shaded}
\begin{Highlighting}[]
\NormalTok{lda\_duas\_dist }\OtherTok{\textless{}{-}} \FunctionTok{lda}\NormalTok{(deslocamento }\SpecialCharTok{\textasciitilde{}}\NormalTok{ distanciaA }\SpecialCharTok{+}\NormalTok{ distanciaF, }\AttributeTok{data =}\NormalTok{ treino)}
\NormalTok{lda\_pred\_duas\_dist }\OtherTok{\textless{}{-}}  \FunctionTok{predict}\NormalTok{(lda\_duas\_dist, }\AttributeTok{newdata =}\NormalTok{ teste)}
\end{Highlighting}
\end{Shaded}

Calculando as taxas de erro dos modelos de distância aberta e com duas
distâncias

\begin{Shaded}
\begin{Highlighting}[]
\NormalTok{(db }\OtherTok{\textless{}{-}}\NormalTok{  teste }\SpecialCharTok{\%\textgreater{}\%}  
   \FunctionTok{mutate}\NormalTok{(}\AttributeTok{deslocamento\_pred\_dist\_aberta =}\NormalTok{ lda\_pred\_dist\_aberta}\SpecialCharTok{$}\NormalTok{class,}
          \AttributeTok{erro\_pred\_dist\_aberta =}\NormalTok{ deslocamento\_pred\_dist\_aberta }\SpecialCharTok{!=}\NormalTok{ deslocamento,}
          \AttributeTok{deslocamento\_pred\_duas\_dist =}\NormalTok{ lda\_pred\_duas\_dist}\SpecialCharTok{$}\NormalTok{class,}
          \AttributeTok{erro\_pred\_duas\_dist =}\NormalTok{ deslocamento\_pred\_duas\_dist }\SpecialCharTok{!=}\NormalTok{ deslocamento,}
\NormalTok{          )}
\NormalTok{ )}
\end{Highlighting}
\end{Shaded}

\begin{verbatim}
## # A tibble: 24 x 7
##    deslocamento distanciaA distanciaF deslocamento_pred_dist_aberta
##           <dbl>      <dbl>      <dbl> <fct>                        
##  1            1        2          1.3 0                            
##  2            0        1.5        2.2 0                            
##  3            0        1.7        1   0                            
##  4            0        1.9        1.4 0                            
##  5            1        2.5        3.1 0                            
##  6            0        1.4        1.5 0                            
##  7            1        2.5        1.8 0                            
##  8            1        2.3        1.6 0                            
##  9            0        1.2        0.4 0                            
## 10            0        1          1.1 0                            
## # i 14 more rows
## # i 3 more variables: erro_pred_dist_aberta <lgl>,
## #   deslocamento_pred_duas_dist <fct>, erro_pred_duas_dist <lgl>
\end{verbatim}

\begin{Shaded}
\begin{Highlighting}[]
\NormalTok{taxa\_erro\_pred\_dist\_aberta }\OtherTok{\textless{}{-}}  \FunctionTok{sum}\NormalTok{(db}\SpecialCharTok{$}\NormalTok{erro\_pred\_dist\_aberta)}\SpecialCharTok{/}\DecValTok{24}
\NormalTok{taxa\_erro\_pred\_duas\_dist }\OtherTok{\textless{}{-}}  \FunctionTok{sum}\NormalTok{(db}\SpecialCharTok{$}\NormalTok{erro\_pred\_duas\_dist)}\SpecialCharTok{/}\DecValTok{24}
\end{Highlighting}
\end{Shaded}

A taxa de erro do modelo com distância aberta é 0.2916667;a taxa de erro
do modelo com duas distâncias é 0.2083333.

Criando a tabela para lda distância aberta:

\begin{Shaded}
\begin{Highlighting}[]
\NormalTok{matriz\_dist\_aberta }\OtherTok{\textless{}{-}} \FunctionTok{table}\NormalTok{(}\AttributeTok{previsao =}\NormalTok{ lda\_pred\_dist\_aberta}\SpecialCharTok{$}\NormalTok{class, }\AttributeTok{observado =}\NormalTok{ teste}\SpecialCharTok{$}\NormalTok{deslocamento) }

\NormalTok{acuracia\_dist\_aberta }\OtherTok{\textless{}{-}} \FunctionTok{sum}\NormalTok{(matriz\_dist\_aberta[}\FunctionTok{row}\NormalTok{(matriz\_dist\_aberta) }\SpecialCharTok{==} \FunctionTok{col}\NormalTok{(matriz\_dist\_aberta)])}\SpecialCharTok{/} \FunctionTok{sum}\NormalTok{(matriz\_dist\_aberta)}
\NormalTok{sensibilidade\_dist\_aberta }\OtherTok{\textless{}{-}}\NormalTok{ matriz\_dist\_aberta[}\DecValTok{1}\NormalTok{,}\DecValTok{1}\NormalTok{]}\SpecialCharTok{/}\FunctionTok{sum}\NormalTok{(matriz\_dist\_aberta[}\DecValTok{1}\NormalTok{,}\DecValTok{1}\NormalTok{],matriz\_dist\_aberta[}\DecValTok{2}\NormalTok{,}\DecValTok{1}\NormalTok{])}
\NormalTok{especificidade\_dist\_aberta }\OtherTok{\textless{}{-}}\NormalTok{ matriz\_dist\_aberta[}\DecValTok{2}\NormalTok{,}\DecValTok{2}\NormalTok{]}\SpecialCharTok{/}\FunctionTok{sum}\NormalTok{(matriz\_dist\_aberta[}\DecValTok{1}\NormalTok{,}\DecValTok{2}\NormalTok{],matriz\_dist\_aberta[}\DecValTok{2}\NormalTok{,}\DecValTok{2}\NormalTok{])}
\NormalTok{matriz\_dist\_aberta}
\end{Highlighting}
\end{Shaded}

\begin{verbatim}
##         observado
## previsao  0  1
##        0 14  7
##        1  0  3
\end{verbatim}

A acurácia do modelo com distância aberta é 0.7083333; a sensibilidade
do modelo com distância aberta é 1; a especificidade do modelo com
distância aberta é 0.3.

Criando a tabela para lda de duas distâncias:

\begin{Shaded}
\begin{Highlighting}[]
\NormalTok{matriz\_duas\_dist }\OtherTok{\textless{}{-}} \FunctionTok{table}\NormalTok{(}\AttributeTok{previsao =}\NormalTok{ lda\_pred\_duas\_dist}\SpecialCharTok{$}\NormalTok{class,}\AttributeTok{observado =}\NormalTok{ teste}\SpecialCharTok{$}\NormalTok{deslocamento) }

\NormalTok{acuracia\_duas\_dist }\OtherTok{\textless{}{-}} \FunctionTok{sum}\NormalTok{(matriz\_duas\_dist[}\FunctionTok{row}\NormalTok{(matriz\_duas\_dist) }\SpecialCharTok{==} \FunctionTok{col}\NormalTok{(matriz\_duas\_dist)])}\SpecialCharTok{/} \FunctionTok{sum}\NormalTok{(matriz\_duas\_dist)}

\NormalTok{sensibilidade\_duas\_dist}\OtherTok{\textless{}{-}}\NormalTok{ matriz\_duas\_dist[}\DecValTok{1}\NormalTok{,}\DecValTok{1}\NormalTok{]}\SpecialCharTok{/}\FunctionTok{sum}\NormalTok{(matriz\_duas\_dist[}\FunctionTok{row}\NormalTok{(matriz\_duas\_dist) }\SpecialCharTok{==} \FunctionTok{col}\NormalTok{(matriz\_duas\_dist)])}
\NormalTok{especificidade\_duas\_dist }\OtherTok{\textless{}{-}}\NormalTok{ matriz\_duas\_dist[}\DecValTok{2}\NormalTok{,}\DecValTok{2}\NormalTok{]}\SpecialCharTok{/}\FunctionTok{sum}\NormalTok{(matriz\_duas\_dist[}\DecValTok{1}\NormalTok{,}\DecValTok{2}\NormalTok{],matriz\_duas\_dist[}\DecValTok{2}\NormalTok{,}\DecValTok{2}\NormalTok{])}
\NormalTok{matriz\_duas\_dist}
\end{Highlighting}
\end{Shaded}

\begin{verbatim}
##         observado
## previsao  0  1
##        0 14  5
##        1  0  5
\end{verbatim}

A acurácia do modelo com duas distâncias é 0.7916667; a sensibilidade do
modelo com distância aberta é 0.7368421; a especificidade do modelo com
distância aberta é 0.5.

Pelo critério de acúracia, o modelo de duas distâncias é melhor

\hypertarget{exeruxedcio-3}{%
\section{Exerício 3}\label{exeruxedcio-3}}

Use o mesmo conjunto de dados do problema anterior e distância aberta
como variável preditora. Use LOOCV e o classificador KNN, com vizinhos
mais próximos de 1 a 5.

\hypertarget{reposta}{%
\subsubsection{Reposta:}\label{reposta}}

\begin{Shaded}
\begin{Highlighting}[]
\CommentTok{\# ajustes na base para estimação}

\NormalTok{disco }\OtherTok{\textless{}{-}}\NormalTok{ disco }\SpecialCharTok{\%\textgreater{}\%} 
  \FunctionTok{mutate}\NormalTok{(}\AttributeTok{deslocamento =} \FunctionTok{as.factor}\NormalTok{(deslocamento))}

\NormalTok{trControl\_loocv }\OtherTok{\textless{}{-}} \FunctionTok{trainControl}\NormalTok{(}\AttributeTok{method =} \StringTok{"LOOCV"}\NormalTok{)}
\NormalTok{knn\_fit }\OtherTok{\textless{}{-}} \FunctionTok{train}\NormalTok{(deslocamento }\SpecialCharTok{\textasciitilde{}}\NormalTok{ distanciaA, }\AttributeTok{method =} \StringTok{"knn"}\NormalTok{,}
\AttributeTok{tuneGrid =} \FunctionTok{expand.grid}\NormalTok{(}\AttributeTok{k =} \DecValTok{1}\SpecialCharTok{:}\DecValTok{5}\NormalTok{),}
\AttributeTok{trControl =}\NormalTok{ trControl\_loocv, }\AttributeTok{metric=} \StringTok{"Accuracy"}\NormalTok{,}
\AttributeTok{data =}\NormalTok{ disco)}
\NormalTok{knn\_fit}
\end{Highlighting}
\end{Shaded}

\begin{verbatim}
## k-Nearest Neighbors 
## 
## 104 samples
##   1 predictor
##   2 classes: '0', '1' 
## 
## No pre-processing
## Resampling: Leave-One-Out Cross-Validation 
## Summary of sample sizes: 103, 103, 103, 103, 103, 103, ... 
## Resampling results across tuning parameters:
## 
##   k  Accuracy   Kappa    
##   1  0.7980769  0.4699029
##   2  0.7980769  0.4699029
##   3  0.7788462  0.3922764
##   4  0.8461538  0.5618747
##   5  0.8076923  0.4776494
## 
## Accuracy was used to select the optimal model using the largest value.
## The final value used for the model was k = 4.
\end{verbatim}

\hypertarget{item-3a}{%
\subsection{Item 3a}\label{item-3a}}

Qual o melhor classificador baseado na acurácia?

\hypertarget{resposta-3}{%
\subsubsection{Resposta:}\label{resposta-3}}

O melhor classsificador, baseado em acuária, é o modelo knn com k = 4.

\hypertarget{item-3b}{%
\subsection{Item 3b}\label{item-3b}}

Obtenha a matriz de confusão e realize o teste de McNemar.

\hypertarget{resposta-4}{%
\subsubsection{Resposta:}\label{resposta-4}}

\begin{Shaded}
\begin{Highlighting}[]
\NormalTok{predict\_knn }\OtherTok{\textless{}{-}} \FunctionTok{predict}\NormalTok{(knn\_fit)}
\NormalTok{matriz\_confusao\_tudo }\OtherTok{\textless{}{-}} \FunctionTok{confusionMatrix}\NormalTok{(predict\_knn, disco}\SpecialCharTok{$}\NormalTok{deslocamento)}
\NormalTok{matriz\_confusao }\OtherTok{\textless{}{-}}\NormalTok{  matriz\_confusao\_tudo}\SpecialCharTok{$}\NormalTok{table}
\NormalTok{mcnemars\_kkn }\OtherTok{\textless{}{-}}\NormalTok{ matriz\_confusao\_tudo[[}\StringTok{"overall"}\NormalTok{]][[}\StringTok{"McnemarPValue"}\NormalTok{]]}

\NormalTok{matriz\_confusao\_tudo}
\end{Highlighting}
\end{Shaded}

\begin{verbatim}
## Confusion Matrix and Statistics
## 
##           Reference
## Prediction  0  1
##          0 71 12
##          1  4 17
##                                           
##                Accuracy : 0.8462          
##                  95% CI : (0.7622, 0.9094)
##     No Information Rate : 0.7212          
##     P-Value [Acc > NIR] : 0.002035        
##                                           
##                   Kappa : 0.5821          
##                                           
##  Mcnemar's Test P-Value : 0.080118        
##                                           
##             Sensitivity : 0.9467          
##             Specificity : 0.5862          
##          Pos Pred Value : 0.8554          
##          Neg Pred Value : 0.8095          
##              Prevalence : 0.7212          
##          Detection Rate : 0.6827          
##    Detection Prevalence : 0.7981          
##       Balanced Accuracy : 0.7664          
##                                           
##        'Positive' Class : 0               
## 
\end{verbatim}

A matriz de confusão é 71, 4, 12, 17. A estatística do teste de
Mcnemar's é 0.0801183.

\hypertarget{item-3c}{%
\subsection{Item 3c}\label{item-3c}}

Obtenha a sensibilidade e a especificidade e explique seus significados
nesse caso.

\hypertarget{resposta-5}{%
\subsubsection{Resposta:}\label{resposta-5}}

\begin{Shaded}
\begin{Highlighting}[]
\NormalTok{knn\_sensitivity }\OtherTok{\textless{}{-}}\NormalTok{ matriz\_confusao\_tudo}\SpecialCharTok{$}\NormalTok{byClass[}\DecValTok{1}\NormalTok{]}

\NormalTok{knn\_specificity }\OtherTok{\textless{}{-}}\NormalTok{ matriz\_confusao\_tudo}\SpecialCharTok{$}\NormalTok{byClass[}\DecValTok{2}\NormalTok{]}
\end{Highlighting}
\end{Shaded}

A sensibilidade é 0.9466667 e a especificidade é 0.5862069. Aqui a
sensibilidade é uma estimativa das probabilidades de decisões corretas
quando o disco realmente não esta deslocado. A especificidade, por sua
vez, é uma estimativa das probabilidades de decisões corretas quando o
disco realmente está deslocado

\hypertarget{exercuxedcio-4}{%
\section{Exercício 4}\label{exercuxedcio-4}}

O conjunto de dados \texttt{Auto} do pacote \texttt{ISLR} contém as
seguintes variáveis:

\begin{itemize}
\item
  \texttt{mpg}: miles per gallon
\item
  \texttt{cylinders}: Number of cylinders between 4 and 8
\item
  \texttt{displacement}: Engine displacement (cubic inches)
\item
  \texttt{horsepower}: Engine horsepower
\item
  \texttt{weight}: Vehicle weight (lbs.)
\item
  \texttt{acceleration}: Time to accelerate from 0 to 60 mph (sec.)
\item
  \texttt{year}: Model year (modulo 100)
\item
  \texttt{origin}: Origin of car (1. American, 2. European, 3. Japanese)
\item
  \texttt{name}: Vehicle name
\end{itemize}

\hypertarget{item-4a}{%
\subsection{Item 4a}\label{item-4a}}

Divida os dados em conjunto de treinamento (\texttt{S}) e conjunto de
teste (\texttt{T}).

\hypertarget{resposta-6}{%
\subsubsection{Resposta:}\label{resposta-6}}

\begin{Shaded}
\begin{Highlighting}[]
\NormalTok{sample }\OtherTok{\textless{}{-}} \FunctionTok{sample}\NormalTok{(}\FunctionTok{c}\NormalTok{(}\ConstantTok{TRUE}\NormalTok{, }\ConstantTok{FALSE}\NormalTok{), }\FunctionTok{nrow}\NormalTok{(Auto), }\AttributeTok{replace=}\ConstantTok{TRUE}\NormalTok{, }\AttributeTok{prob=}\FunctionTok{c}\NormalTok{(}\FloatTok{0.7}\NormalTok{,}\FloatTok{0.3}\NormalTok{))}
\NormalTok{S  }\OtherTok{\textless{}{-}}\NormalTok{ Auto[sample, ]}
\NormalTok{T   }\OtherTok{\textless{}{-}}\NormalTok{ Auto[}\SpecialCharTok{!}\NormalTok{sample, ]}
\end{Highlighting}
\end{Shaded}

\hypertarget{item-4b}{%
\subsection{Item 4b}\label{item-4b}}

Ajuste um modelo aos dados de \texttt{S} tendo \texttt{horsepower} como
preditor e \texttt{mpg} como resposta. Obtenha os EMQ de treinamento e
faça o diagnóstico do modelo. O que você nota no gráfico dos resíduos
contra valores ajustados? Obtenha o EQM de teste.

\hypertarget{resposta-7}{%
\subsubsection{Resposta:}\label{resposta-7}}

Estimando modelo de EMQ de treinamtno

\begin{Shaded}
\begin{Highlighting}[]
\CommentTok{\# EMQ de treinamento}
\NormalTok{emq\_auto }\OtherTok{\textless{}{-}} \FunctionTok{glm}\NormalTok{(}\AttributeTok{formula =}\NormalTok{ mpg }\SpecialCharTok{\textasciitilde{}}\NormalTok{ horsepower, }\AttributeTok{data =}\NormalTok{ S) }
\FunctionTok{summary}\NormalTok{(emq\_auto)}
\end{Highlighting}
\end{Shaded}

\begin{verbatim}
## 
## Call:
## glm(formula = mpg ~ horsepower, data = S)
## 
## Deviance Residuals: 
##      Min        1Q    Median        3Q       Max  
## -13.6436   -3.1693   -0.3081    2.7162   16.8378  
## 
## Coefficients:
##              Estimate Std. Error t value Pr(>|t|)    
## (Intercept) 40.149014   0.849423   47.27   <2e-16 ***
## horsepower  -0.159797   0.007611  -20.99   <2e-16 ***
## ---
## Signif. codes:  0 '***' 0.001 '**' 0.01 '*' 0.05 '.' 0.1 ' ' 1
## 
## (Dispersion parameter for gaussian family taken to be 23.53215)
## 
##     Null deviance: 16796.2  on 274  degrees of freedom
## Residual deviance:  6424.3  on 273  degrees of freedom
## AIC: 1653
## 
## Number of Fisher Scoring iterations: 2
\end{verbatim}

Criando gráfico dos resíduos contra valores ajustados.

\begin{Shaded}
\begin{Highlighting}[]
\NormalTok{emq\_valores\_ajustados }\OtherTok{\textless{}{-}} \FunctionTok{fitted.values}\NormalTok{(emq\_auto)}
\NormalTok{emq\_residuos }\OtherTok{\textless{}{-}} \FunctionTok{residuals}\NormalTok{(emq\_auto)}

\FunctionTok{plot}\NormalTok{(emq\_valores\_ajustados, emq\_residuos, }
     \AttributeTok{xlab =} \StringTok{"Valores Ajustados"}\NormalTok{, }
     \AttributeTok{ylab =} \StringTok{"Resíduos"}\NormalTok{,}
     \AttributeTok{main =} \StringTok{"Gráfico de Valores Ajustados vs. Resíduos"}\NormalTok{)}
\end{Highlighting}
\end{Shaded}

\includegraphics{prova1_files/figure-latex/unnamed-chunk-17-1.pdf} O
gráfico

calculando EQM de teste:

\begin{Shaded}
\begin{Highlighting}[]
\NormalTok{emq\_auto\_predict\_T }\OtherTok{\textless{}{-}} \FunctionTok{predict}\NormalTok{(emq\_auto, }\AttributeTok{newdata =}\NormalTok{ T)}
\end{Highlighting}
\end{Shaded}

\hypertarget{item-4c}{%
\subsection{Item 4c}\label{item-4c}}

Agora inclua \((\text{horsepower})^2\) no modelo e proceda como no item
(b). Qual modelo você escolheria? Justifique.

\hypertarget{resposta-8}{%
\subsubsection{Resposta:}\label{resposta-8}}

\begin{Shaded}
\begin{Highlighting}[]
\NormalTok{emq\_auto\_quadrado }\OtherTok{\textless{}{-}} \FunctionTok{glm}\NormalTok{(}\AttributeTok{formula =}\NormalTok{ mpg }\SpecialCharTok{\textasciitilde{}}\NormalTok{ horsepower }\SpecialCharTok{+}\NormalTok{ horsepower}\SpecialCharTok{\^{}}\DecValTok{2}\NormalTok{, }\AttributeTok{data =}\NormalTok{ S) }
\FunctionTok{summary}\NormalTok{(emq\_auto\_quadrado)}
\end{Highlighting}
\end{Shaded}

\begin{verbatim}
## 
## Call:
## glm(formula = mpg ~ horsepower + horsepower^2, data = S)
## 
## Deviance Residuals: 
##      Min        1Q    Median        3Q       Max  
## -13.6436   -3.1693   -0.3081    2.7162   16.8378  
## 
## Coefficients:
##              Estimate Std. Error t value Pr(>|t|)    
## (Intercept) 40.149014   0.849423   47.27   <2e-16 ***
## horsepower  -0.159797   0.007611  -20.99   <2e-16 ***
## ---
## Signif. codes:  0 '***' 0.001 '**' 0.01 '*' 0.05 '.' 0.1 ' ' 1
## 
## (Dispersion parameter for gaussian family taken to be 23.53215)
## 
##     Null deviance: 16796.2  on 274  degrees of freedom
## Residual deviance:  6424.3  on 273  degrees of freedom
## AIC: 1653
## 
## Number of Fisher Scoring iterations: 2
\end{verbatim}

\hypertarget{item-4d}{%
\subsection{Item 4d}\label{item-4d}}

Ajuste um modelo de regressão ridge aos dados de \(S\), tendo
\texttt{mpg} com resposta e \texttt{displacement}, \texttt{horsepower},
\texttt{weight} e \texttt{acceleration} como preditores, com \(\lambda\)
escolhido por VC. Obtenha o EQM de teste.

\hypertarget{resposta-9}{%
\subsubsection{Resposta:}\label{resposta-9}}

Transformando dataframe em matriz

\begin{Shaded}
\begin{Highlighting}[]
\NormalTok{X\_linha }\OtherTok{\textless{}{-}}\NormalTok{ S[,}\DecValTok{3}\SpecialCharTok{:}\DecValTok{6}\NormalTok{]}
\NormalTok{X\_linha\_matriz }\OtherTok{\textless{}{-}} \FunctionTok{as.matrix}\NormalTok{(X\_linha)}
\NormalTok{Y\_linha }\OtherTok{\textless{}{-}} \FunctionTok{as.matrix}\NormalTok{(S[,}\DecValTok{1}\NormalTok{])}
\end{Highlighting}
\end{Shaded}

Estimando o modelo Ridge, plotando o gráfico para efeito da avaliação do
efeito do coeficiente de regularização e imprimindo coeficientes.

\begin{Shaded}
\begin{Highlighting}[]
\CommentTok{\# estimando o modelo ridge}
\NormalTok{emq\_auto\_ridge }\OtherTok{\textless{}{-}} \FunctionTok{cv.glmnet}\NormalTok{(X\_linha\_matriz, Y\_linha, }\AttributeTok{alpha =} \DecValTok{0}\NormalTok{) }
\CommentTok{\# plot(emq\_auto\_ridge)}
\FunctionTok{coef}\NormalTok{(emq\_auto\_ridge, }\AttributeTok{s =} \StringTok{"lambda.min"}\NormalTok{)}
\end{Highlighting}
\end{Shaded}

\begin{verbatim}
## 5 x 1 sparse Matrix of class "dgCMatrix"
##                        s1
## (Intercept)  43.455710885
## displacement -0.017688879
## horsepower   -0.047802163
## weight       -0.003559223
## acceleration -0.062057243
\end{verbatim}

\begin{Shaded}
\begin{Highlighting}[]
\NormalTok{lambda\_vc }\OtherTok{\textless{}{-}}\NormalTok{ emq\_auto\_ridge}\SpecialCharTok{$}\NormalTok{lambda.min}
\end{Highlighting}
\end{Shaded}

O lambda escolhido por VC é 0.6505238.

calculando o EQM.

\begin{Shaded}
\begin{Highlighting}[]
\NormalTok{emq\_auto\_ridge\_predict }\OtherTok{\textless{}{-}} \FunctionTok{predict}\NormalTok{(emq\_auto\_ridge, X\_linha\_matriz, }\AttributeTok{s =} \StringTok{"lambda.min"}\NormalTok{)}
\NormalTok{emq\_auto\_ridge\_eqm }\OtherTok{\textless{}{-}} \FunctionTok{sqrt}\NormalTok{(emq\_auto\_ridge}\SpecialCharTok{$}\NormalTok{cvm[emq\_auto\_ridge}\SpecialCharTok{$}\NormalTok{lambda }\SpecialCharTok{==}\NormalTok{ emq\_auto\_ridge}\SpecialCharTok{$}\NormalTok{lambda.min])}
\NormalTok{emq\_auto\_ridge\_r2 }\OtherTok{\textless{}{-}}\NormalTok{ emq\_auto\_ridge}\SpecialCharTok{$}\NormalTok{dev.ratio}
\end{Highlighting}
\end{Shaded}

Temos que o EQM é 4.2649823

\hypertarget{item-4e}{%
\subsection{Item 4e}\label{item-4e}}

Ajuste um modelo de regressão \texttt{lasso} e proceda como em (d).
Quais coeficientes foram zerados? Comente sobre os resultados obtidos em
(d) e (e), baseados no \(R^2\) e \(EQMI\).

\hypertarget{resposta-10}{%
\subsubsection{Resposta}\label{resposta-10}}

Estimando o modelo de regerssão lasso e imprimindo os coeficientes.

\begin{Shaded}
\begin{Highlighting}[]
\NormalTok{emq\_auto\_lasso }\OtherTok{=} \FunctionTok{cv.glmnet}\NormalTok{(X\_linha\_matriz, Y\_linha, }\AttributeTok{alpha =} \DecValTok{1}\NormalTok{)}
\FunctionTok{coef}\NormalTok{(emq\_auto\_lasso, }\AttributeTok{s =} \StringTok{"lambda.min"}\NormalTok{)}
\end{Highlighting}
\end{Shaded}

\begin{verbatim}
## 5 x 1 sparse Matrix of class "dgCMatrix"
##                        s1
## (Intercept)  43.737436088
## displacement -0.011208999
## horsepower   -0.040513292
## weight       -0.004654042
## acceleration  .
\end{verbatim}

Como se pode notar, o coeficiente de acceleration foi zerado.

Calculando \(R^2\) e \(EQMI\) do Lasso.

\begin{Shaded}
\begin{Highlighting}[]
\NormalTok{emq\_auto\_lasso\_eqm }\OtherTok{\textless{}{-}} \FunctionTok{sqrt}\NormalTok{(emq\_auto\_lasso}\SpecialCharTok{$}\NormalTok{cvm[emq\_auto\_lasso}\SpecialCharTok{$}\NormalTok{lambda }\SpecialCharTok{==}\NormalTok{ emq\_auto\_lasso}\SpecialCharTok{$}\NormalTok{lambda.min])}
\NormalTok{emq\_auto\_lasso\_r2 }\OtherTok{\textless{}{-}}\NormalTok{ emq\_auto\_lasso}\SpecialCharTok{$}\NormalTok{dev.ratio}
\end{Highlighting}
\end{Shaded}


\end{document}
